\subsection{Result on Anomaly-ShapeNet}
\label{sec:results-shapenet}

Table~\ref{tab:ShapeNet} and Figure~\ref{fig:shapenet} summarize our quantitative and qualitative results on Anomaly-ShapeNet. Our method attains a point-level AUROC of 91.2\% and point-level AUPR of 38.7\%, which improves over the strongest baseline, Group3AD, by 6.6\% and 13.3\%, respectively. At the object level our AUROC is 86.8\% and AUPR is 98.7\%, exceeding Group3AD by 5.4\% and 3.4\%. These accuracy gains are achieved alongside a large runtime advantage: our default configuration runs at 17.31 FPS on the RTX-3090 workstation, which is substantially faster than the competing methods listed in Table~\ref{tab:ShapeNet}.

BTF(Raw) refers to the use of only the raw 3D coordinate features (x, y, z) within the Back-to-Front (BTF) framework \cite{horwitz2023back}. In contrast, BTF(FPFH) augments the same pipeline with Fast Point Feature Histograms (FPFH) \cite{rusu2009fast}. The entries M3DM(PointMAE) and M3DM(PointBERT) correspond to the model \cite{wang2023multimodal} configured to ignore its RGB branch and instead extract point cloud features with PointMAE \cite{pang2022masked} or PointBERT \cite{yu2022point}, respectively. For PatchCore variants, PatchCore(FPFH) replaces the usual ResNet-based feature extractor with FPFH descriptors \cite{rusu2009fast} before feeding them into the PatchCore anomaly scoring pipeline \cite{roth2022towards}. PatchCore(FPFH+Raw) further concatenates the raw spatial coordinates to each FPFH feature vector, and PatchCore(PointMAE) uses the PointMAE network \cite{pang2022masked} as the backbone feature extractor within the PatchCore framework.

The observed improvements can be traced to three design choices. First, multi-scale token fusion with the Mamba adapter aggregates complementary cues across Transformer layers so that subtle local deviations are evaluated in their broader geometric context. Defects such as bulges and concavities manifest across receptive fields and benefit from the cross-layer evidence that Mamba supplies. Second, the Geometric Semantic-Aware Sorter (GSAS) produces a geometry-preserving ordering that enables state-space fusion to propagate information along coherent surface trajectories rather than arbitrary token sequences. This geometric consistency reduces spurious contextual mixing and sharpens localization. Third, the selective anomalous feature generator provides localized, feature-space supervision that teaches the discriminator to attend to sparse, realistic deviations without corrupting global shape priors. Qualitatively, these components combine to produce compact heatmaps with low background noise for medium and large localized defects, as shown in Figure~\ref{fig:shapenet}.

Failure modes are consistent with expectations for point-based pipelines. Very thin cracks that remove only a few points remain challenging and reduce AUPR because a small number of mis-scored points strongly affects precision. High-curvature ornamental details can sometimes be mistaken for defects, which suggests that future work could incorporate curvature-aware post-processing or augment the anomalous generator with structure-preserving perturbations. Overall, the results on Anomaly-ShapeNet demonstrate that the combination of GSAS, Mamba fusion, and sparse feature-space supervision yields both state-of-the-art detection and a deployment-friendly runtime.

\subsection{Result on Real3D-AD}
\label{sec:results-real3d}

Table~\ref{tab:Real3D} and Figure~\ref{fig:real3d} report performance on Real3D-AD, where scans are high density and defects arise under realistic sensor conditions. We compare with the same set of state-of-the-art methods used for Anomaly-ShapeNet. Our method achieves a point-level AUROC of 76.6\% and point-level AUPR of 19.7\%, improving over Group3AD by 2.8\% and 5.8\%, respectively. At the object level we obtain an AUROC of 78.4\% and an AUPR of 77.7\%, improving over Group3AD by 3.1\% and 3.4\%. As on the synthetic benchmark, our approach maintains a large throughput advantage with 17.31 FPS, which supports practical deployment in industrial inspection pipelines.

Real3D-AD differs from synthetic benchmarks in three ways that highlight the strengths of our design. First, defects in real scans are often subtle and must be discriminated from sensor noise and varying point density. Mamba fusion aggregates multi-layer signals so that local irregularities are evaluated with respect to global part geometry, which reduces false positives caused by sensor artifacts. Second, GSAS ensures that the fusion operates on a surface-aware ordering, which is important in dense, complex scans to prevent unrelated surface regions from contaminating contextual cues. Third, the selective anomalous feature generator produces spatially sparse supervisory signals that encourage sensitivity to small but consistent deviations, which is useful when prototype sets are limited and supervised defect examples are scarce.

Qualitatively, our method reliably highlights dents, missing material, and other geometric defects with high contrast while keeping false detections low in benign regions. Limitations include reduced sensitivity to appearance-only anomalies that do not affect geometry and occasional loss of localization precision in heavily occluded or extremely unevenly sampled areas. Practical mitigations include fusing appearance or reflectance channels when available and applying local density normalization during preprocessing.

In summary, results on Real3D-AD corroborate the conclusions from the synthetic benchmark. The combination of geometry-preserving ordering, linear-time cross-layer fusion, and sparse feature-space supervision improves both detection and localization robustness while offering a favorable runtime and memory profile for industrial applications.
