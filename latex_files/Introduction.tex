\section{Introduction}

Industrial anomaly detection in the two-dimensional (2D) image domain has matured substantially over the last decade and provides a useful baseline for inspection tasks \cite{samrouth2025dual, casas2024comparative, liu2025improved}. Supervised and semi-supervised methods based on convolutional neural networks and transformer architectures achieve high detection accuracy when sufficient annotated defects are available \cite{venkataramanan2020attention,ding2022catching}. Unsupervised approaches, which train exclusively on defect-free images, have proven particularly practical: reconstruction-based models (autoencoders, GANs, diffusion systems) and feature-space methods (teacher-student distillation, memory banks, nearest-neighbor scoring) can localize diverse defect types without exhaustive annotations \cite{shi2021dfr,hou2021divide,zavrtanik2021draem,zavrtanik2022dsr,de2022masked,yao2023focus,defard2021padim,bergmann2020uninformed}. A complementary line of work synthesizes pseudo-anomalies in pixel or feature space (e.g., CutPaste, DRAEM, SimpleNet) to enable discriminative training from only normal samples \cite{li2021cutpaste,zavrtanik2021draem,liu2023simplenet}. These approaches benefit from dense photometric cues, regular grid structure, and a rich ecosystem of pretrained image encoders, which together simplify representation learning and spatial localization. Despite their practical success, purely 2D approaches have intrinsic limitations when applied to geometric defect detection. Photometric cues are unavailable or uninformative for many surface faults (e.g., subtle dents, thin cracks, or missing material) that primarily affect shape rather than appearance. Furthermore, 2D projections collapse 3D geometry and can hide defects through viewpoint or occlusion; compensating for this requires multi-view capture and careful alignment, which increases acquisition and processing complexity. Finally, feature distributions learned on natural images transfer imperfectly to industrial scans, producing domain shifts that reduce sensitivity to subtle, geometry-driven anomalies.

These limitations motivate direct anomaly analysis in three-dimensional (3D) data. Depth sensors and structured-light scanners provide point clouds and range maps that expose surface geometry, sampling density, and local curvature, which are essential for detecting geometry-only defects. Recent 3D methods exploit these cues using several strategies. Multimodal fusion approaches combine geometric descriptors with image-derived semantics to leverage complementary information\cite{cao2024complementary,wang2023multimodal,rudolph2023asymmetric}. Purely geometric pipelines operate directly on point clouds via teacher-student distillation, memory banks, registration, and prototype clustering to represent normal surface patterns \cite{bergmann2023anomaly,liu2023real3d,zhu2024towards}. Generative and self-supervised reconstruction models (masked reconstruction, diffusion-based denoising) learn to map anomalous inputs back to normal shapes and score deviations at the point level \cite{li2024towards,zhou2024r3d}. These directions demonstrate that 3D-specific cues substantially improve defect localization for shape-centric faults. However, existing 3D solutions exhibit several recurring limitations that impede robust, high-throughput deployment. First, many methods rely on handcrafted descriptors, large memory banks, costly registration or clustering procedures, or iterative generative inference; each increases runtime, memory footprint, or system complexity and thus complicates integration into production pipelines \cite{liu2023real3d,zhu2024towards}. Second, point clouds are unordered, irregularly sampled, and noisy, yet several approaches impose brittle, non-differentiable orderings or heuristic traversals to linearize geometry for sequence models; such orderings can mix distant surface regions and degrade context propagation \cite{liang2024pointmamba}. Third, reconstruction or input-space synthesis is sensitive to sampling artifacts and may inadvertently reconstruct defects, reducing anomaly contrast \cite{li2024towards,zhou2024r3d}. Fourth, the scarcity and narrow diversity of publicly available 3D anomaly benchmarks limit supervised fine-tuning and motivate synthetic or feature-space augmentation strategies, but the design of realistic pseudo-anomalies in 3D feature spaces remains under-explored. Finally, while large pretrained 3D backbones (e.g., masked point-cloud transformers) offer strong representations, most 3D anomaly methods either fine-tune them (incurring cost) or ignore cross-layer token structure (losing multi-scale cues); lightweight, differentiable adapters that preserve geometric locality are largely missing.

To address these challenges we propose a compact, practical framework\footnote{Code and dataset: \url{https://github.com/hoangcuongbk80/Mamba3dAD}} that combines geometry-preserving ordering, efficient sequential fusion, and localized discriminative scoring while leveraging large pretrained 3D Transformers with minimal backbone adaptation. First, we introduce a Geometric Semantic-Aware Sorter (GSAS), a differentiable soft-permutation that converts unordered multi-layer per-patch tokens into a spatially coherent ordered sequence; GSAS preserves local surface adjacency and produces contiguous surface trajectories that expose geometric structure to downstream sequence models. Second, we present Mamba, a linear-time state-space adapter that fuses the GSAS-ordered slots and propagates long-range, cross-layer context with minimal parameter and compute overhead, enabling contextual reasoning comparable to attention at substantially lower cost \cite{gu2021efficiently,gu2020hippo}. Third, an attention-based discriminator operates on the fused slot sequence and produces slot-level anomaly logits that are reprojected to points for dense localization. To train the discriminator without dense real-defect labels, we synthesize sparse pseudo-anomalies directly in the fused feature space; this feature-space corruption strategy yields localized supervisory signals that focus the model on discriminative deviations while preserving global shape priors. The resulting pipeline is compatible with frozen pretrained point cloud encoders, requires only lightweight adapter parameters, and scales to high-resolution scans.

Our method delivers substantial empirical gains while remaining computationally practical. On the synthetic Anomaly-ShapeNet benchmark we achieve a point-level AUROC of 91.2\% and AUPR of 38.7\%, and an object-level AUROC/AUPR of 86.8\%/98.7\%, improving over the strongest prior competitor (Group3AD) by 6.6\%  in point AUROC and 13.3\% in point AUPR. On the high-fidelity Real3D-AD scans we obtain a point-level AUROC of 76.6\% and AUPR of 19.7\%, and an object-level AUROC/AUPR of 78.4\%/77.7\%, exceeding Group3AD by 2.8\% and 5.8\% in point AUROC and AUPR, respectively. These gains are accompanied by practical runtime: the default configuration processes scans at roughly 17.3\,FPS, which is several times faster than reported runtimes of competing methods while preserving or improving localization accuracy. The results demonstrate that differentiable geometry-aware ordering plus efficient sequence fusion enables both state-of-the-art accuracy and the latency properties required for industrial inspection.

The main contributions of this paper are:
\begin{itemize}
  \item We propose a differentiable Geometric Semantic-Aware Sorter (GSAS) that produces spatially coherent soft-orderings of multi-layer 3D tokens, enabling sequence-modeling without sacrificing local surface fidelity.
  \item We introduce Mamba, a linear-time state-space adapter tailored to fuse ordered slot features and propagate long-range geometric context with low runtime and parameter overhead.
  \item We design an attention-based discriminator trained with sparse feature-space pseudo-anomalies to provide accurate point-level localization without requiring dense defect annotations.
  \item We validate the full pipeline on synthetic and real 3D anomaly benchmarks, demonstrating improved point- and object-level AUROC/AUPR over recent methods while maintaining practical inference speed suitable for deployment.
\end{itemize}